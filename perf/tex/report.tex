\documentclass{article}
\begin{document}
\author {
  Rohit Kumar 
  \\ UCLA 
  \\ \textsl{rohitkk@cs.ucla.edu}
  \and
  Madhuri Venkatesh 
  \\ UCLA 
  \\ \textsl{madhuri@cs.ucla.edu}
  \and
  \textsl{Supervised By:}
  \\Eddie Kohler 
  \\ UCLA 
  \\ \textsl{kohler@cs.ucla.edu}
}
\title{Cache Conscious Radix Tree Design}
\maketitle
\newcounter{hyp-count}
\addtocounter{hyp-count}{1}
\newcommand{\hypothesis}[1]{\paragraph{Hypothesis \arabic{hyp-count}:}
  \textsl{#1} \addtocounter{hyp-count}{1}}
\section{ABSTRACT}
\section{INTRODUCTION}
\section{RELATED WORK}
\section{EVALUATION}

\subsection{Changing Branching Factors}

\hypothesis{For routing tables, which are dominated by longer prefixes, a
higher level-1 branching factor, will lead to lower cache misses.} 

\subsection{Using a Custom Allocator}
\hypothesis{Ensuring that all Radix nodes are allocated from a
  large contigous block of memory, will lead to lower cache misses.}

\subsection{Removing Derivable Fields}
\hypothesis{Removing any derivable fields from the Radix node, will
  lead to lower cache misses}

\subsection{Using a Static Lookup Table}
\hypothesis{Using a static array to lookup the values of derivable
  fields will cause an improvement in performance}

\subsection{Prefetching Child Nodes}
\hypothesis{Once we know the next child node which is going to be
  accessed by the lookup, using software prefetch will help us reduce
  cache misses}

\subsection{Skip Nodes}
\subsection{Detecting Null Children With Bit Value}
\subsection{Memalign}

\section{CONCLUSION}
\section{FUTURE WORK}


\end{document}
